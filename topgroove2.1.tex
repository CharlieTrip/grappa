\documentclass[10pt,a4paper]{article}

% Matematica
\usepackage{amsmath,amsthm,amssymb,amsfonts, mathrsfs}
% Accenti
\usepackage[utf8]{inputenc}
\usepackage[italian, english]{babel}
% Altro
\usepackage{anysize, hyperref}

% Teoremi e simili
\newtheoremstyle{icorollary}{}{}{}{}{\itshape}{.---}{0pt}{#1}
\newtheoremstyle{numcorollary}{}{}{}{}{\itshape}{.}{ }{#1\if!#3!\else\ \fi\thmnote{#3}}
\newtheoremstyle{idefinition}{}{}{}{}{\bfseries}{.---}{0pt}{}
\newtheoremstyle{ilemma}{}{}{\itshape}{}{\bfseries}{.---}{0pt}{#1\if!#3!\else\ \fi\thmnote{#3}}
\newtheoremstyle{iother}{}{}{\itshape}{}{\bfseries}{.---}{0pt}{\thmnote{#3}}
\theoremstyle{ilemma}
\newtheorem*{lemma}{Lemma}
\theoremstyle{plain}
\newtheorem{theorem}{Theorem}
\theoremstyle{plain}
\newtheorem*{proposition}{Proposizione}
\theoremstyle{iother}
\newtheorem{other}{}
\theoremstyle{icorollary}
\newtheorem{corollary}{Corollary}
\theoremstyle{numcorollary}
\newtheorem{ncorollary}{Corollary}
\theoremstyle{plain}
\newtheorem*{definition}{Definizione}
\newtheorem*{definitions}{Definizioni}
\newtheorem*{defnorder}{Definition of Order}

\renewcommand{\qedsymbol}{$\blacksquare$}

\title{Topology Groove (pt. II.1)}

\begin{document}
\maketitle

Definiamo i connessi per contrasto:
\begin{definitions}
  Sia $(X,\mathcal T)$ spazio topologico, allora
  \begin{itemize}
  \item[(i)] Diciamo che $A,B\in\mathcal T$ costituiscono una \emph{separazione}
    di $X$ se $A,B\neq\emptyset=A\cap B$, $A\cup B=X$;
  \item[(ii)] Diciamo che $X$ \`e connesso se $\nexists A,B\in\mathcal T$ che
    costituiscano una separazione.
  \end{itemize}
\end{definitions}
Patologie: topologia discreta e topologia banale. Aperti nella topologia
indotta.\newline
Fatti utili:
\begin{enumerate}
\item $X$ \`e connesso se e solo se $X,\emptyset$ sono gli unici insiemi aperti
  e chiusi in $(X,\mathcal T)$;
\item $A,B\subset Y\subset (X,\mathcal T)$ costituiscono una separazione di $Y$
  se e solo se $A,B\neq\emptyset=A\cap B$, $A\cup B=Y$ e nessuno dei due
  contiene punti di accumulazione dell'altro.
\end{enumerate}
Come si fa a costruire connessi da connessi? Premettiamo un lemma.
\begin{lemma}
  Sia $(X,\mathcal T)$ spazio topologico, $\{A_i\}_{i\in I}$ una famiglia di
  sottoinsiemi connessi di $X$. Allora
  \begin{itemize}
  \item[(i)] Se $A_j$, $A_k$ costituiscono una separazione di $X$, $A_l$
    connesso $\Rightarrow$ $A_l$ \`e contenuto solo in uno dei due.
  \item[(ii)] $\cap_{i\in I}\{A_i\}\neq\emptyset\Rightarrow \cup_{i\in
      I}\{A_i\}$ connesso.
  \end{itemize}
\end{lemma}
\begin{proof}[dim.] Ad (i). Consideriamo la topologia indotta su $A_l$ e
  osserviamo che $A_j\cap A_l$ e $A_k\cap A_l$ sono aperti. Del resto sono
  disgiunti, e uniti danno $A_k$: se uno dei due non fosse vuoto, ne
  costituirebbero una separazione.\newline
  Ad (ii), per assurdo. Fissato $p\in\cap_{i\in I}\{A_i\}$, separiamo
  $\cup_{i\in I}\{A_i\}$ in $C$ e $D$. Non perdo di generalit\`a dicendo $p\in
  C$, del resto $p\in A_i$ per ogni $i$, e gli $A_i$ sono connessi: il risultato
  $(i)$ mi fa concludere $A_i\subset C\ \forall i$, allora $D=\emptyset$.
\end{proof}
Ecco due risultati interessanti:
\begin{proposition} Siano $A,B\subset (X,\mathcal T)$, con $A$ connesso e
$A\subset B\subset\bar A$. Allora $B$ \`e connesso.
\end{proposition}
\begin{proof}[dim.]
  Per assurdo. Separiamo $B$ in $C$ e $D$, da $(i)$ del lemma concludiamo
  $A\subset C$ senza perdita di generalit\`a, segue $\bar A\subset\bar C$, segue
  $B\subset\bar C = C$, segue $B\cap D =\emptyset$.
\end{proof}
\begin{proposition}
  Sia $f:X\to Y$ suriettiva e continua. Se $X$ \`e connesso, anche $Y$ lo \`e.
\end{proposition}
\begin{proof}
  Per assurdo. Separiamo $Y$ in $A$ e $B$, per continuit\`a $f^{-1}(A)$ e
  $f^{-1}(B)$ sono aperti, ed \`e immediato verificare $f^{-1}(A)\cap
  f^{-1}(B)=\emptyset$; del resto per suriettivit\`a nessuno dei due \`e vuoto,
  e uniti danno $X$.
\end{proof} Da quest'ultima proposizione si evince che la connessione \`e una
propriet\`a topologica.\newline \`E comodo introdurre un concetto di connessione
per archi:
\begin{definition}
  Sia $(X, \mathcal T)$ spazio topologico. Dati $x$, $y\in X$ diciamo che esiste
  un cammino che connette $x$ a $y$ se $\exists f:[a,b]\to X$ continua e tale
  che $f(a)=x$, $f(b)=y$. Diciamo che $X$ \`e connesso per archi se questo vale
  per qualsiasi coppia di punti di $X$.
\end{definition} Tipico esempio (senza dimostrazione...) di insieme connesso ma
non connesso per archi.\newline Topologia indotta da strumenti di misura e
componenti connesse (per archi) di un insieme.
\begin{definition}Sia $(X,\mathcal T)$ spazio topologico, allora $\forall x,y\in
X$ diciamo
  \begin{enumerate}
  \item $x\sim y$ se $\exists Y\subset X$ che sia connesso e tale che $x$, $y\in
  Y$;
  \item $x\sim_a y$ se $\exists Y\subset X$ che sia connesso per archi e tale
  che $x$, $y\in Y$;
  \end{enumerate} sia $\sim$ che $\sim_a$ sono relazioni di equivalenza:
  verificare riflessivit\`a, simmetria, e transitivit\`a. Per la transitivit\`a
  di $\sim_a$ is utilizza il pasting lemma!
\end{definition} Le classi di equivalenza date da $\sim$ e $\sim_a$ sono dette
\emph{componenti} (arcate) dello spazio $X$.\newline Vale il seguente risultato:
\begin{proposition}
  Ogni componente $C$ di $X$ \`e un suo sottospazio connesso. Le componenti di
  $X$ sono a due a due disgiunte e la loro unione \`e pari a $X$, inoltre ogni
  altro sottospazio connesso di $X$ ($\neq\emptyset$) interseca una sola
  componente.
\end{proposition}
\begin{proof}[dim.]
  Che le componenti siano disgiunte discende direttamente dal fatto che sono
  classi di equivalenza di $\sim$, per la stessa ragione osserviamo che la loro
  unione deve essere $X$. L'ultima affermazione possiamo dimostrarla per
  assurdo: se intersecasse pi\`u d'una componente saremmo in grado di prendere
  un elemento $x_1$ dalla prima intersezione e un $x_2$ dalla seconda per
  concludere $x_1\sim x_2$.\newline Per provare che ogni componente C \`e
  connessa fissiamo $x\in C$ osserviamo che $\forall y_i\in C\exists Y_i\subset
  X$ connesso, ma per quanto appena provato vale $Y_i\subset X$. Ne concludiamo
  $C = \cup_{y_i\in C}Y_i$, e dal punto $(ii)$ del lemma segue la tesi.
\end{proof}

Come ultimo risultato si vorrebbe istituire una qualche relazione
tra componenti e componenti arcate. Per farlo introduciamo una nozione di
connessione locale:
\begin{definition}
  Uno spazio $X$ si dice localmente connesso (per archi) in $x\in X$ se per ogni
  intorno $U$ di $x$ esiste un intorno $V\subset U$ di $x$ che sia connesso (per
  archi). Se questo \`e vero $\forall x\in X$ diciamo che lo spazio \`e
  localmente connesso (per archi).
\end{definition}

Nota: connessione e connessione locale sono concetti slegati. Esempi:
$[-1,0)\cup(0,1]$, seno del topologo (quello di prima).

C'\`e un criterio abbastanza comodo per determinare se uno spazio \`e localmente
connesso:
\begin{proposition}
  Uno spazio $X$ \`e localmente connesso se e solo se per ogni aperto $U$ di $X$
  ogni componente di $U$ \`e aperta.
\end{proposition}
\begin{proof}[dim.]
  ($\Rightarrow$) Sia $U$ aperto in $X$, $C$ una componente di $U$ e $x\in
  C$. Per locale connessione troviamo $V\subset U$ connesso e intorno di $x$;
  per la proposizione precedente $V\subset C$. Allora $C$ \`e aperto, per
  definizione.

  ($\Leftarrow$) Sia $U\subset X$ intorno di qualche $x\in X$, consideriamo la
  componente di $U$ che contiene $x$; tale componente \`e connessa, ed \`e
  aperta per ipotesi (in $X$, dunque in $U$).
\end{proof}

La stessa proposizione, con ovvie modifiche, \`e valida per spazi
localmente connessi.\newline Proviamo un ultimo risultato:
\begin{proposition}
  Sia $(X,\mathcal T)$ spazio topologico, allora ogni componente arcata di $X$
  \`e contenuta in una sua componente (non arcata); se $X$ \`e localmente
  connesso per archi, componenti e componenti arcate coincidono.
\end{proposition}
\begin{proof}[dim.]
  La prima asserzione \`e evidente osservando che le componenti arcate sono
  connesse (non dimostrato, ma del tutto analogo ad un risultato provato
  poc'anzi), e in quanto tali contenute in una componente non arcata.\newline
  Tale inclusione rimane vera anche nella situazione descritta nel secondo caso,
  e ci rimane da dimostrare l'inclusione in senso opposto. Per dimostrarlo,
  consideriamo una componente arcata $P\subset C$ e supponiamo per assurdo
  $P\subsetneq C$; allora per ogni $x\in (C\setminus P)$ consideriamo la
  componente arcata $Q_x\ :\ x\in Q_x$.\newline Vale evidentemente
  $C=P\cup_{x\in C\setminus P}Q_x$, per locale connessione arcata tutte le $Q_x$
  sono aperte, cos\`i come $P$: ma $C$ \`e connessa e $P\cap(\cup_{x\in
  C\setminus P}Q_x)=\emptyset$ (verificare intuitivamente, vero per la prop.ne
  non dimostrata), allora deve essere $\cup_{x\in C\setminus P}Q_x=\emptyset$.

\end{proof}
\end{document}